\documentclass[11pt,letterpaper]{article}                  % Define document class

%\usepackage{amsfonts} % For \mathbb command
%\usepackage[top=1in, bottom=1.25in, left=1in, right=1in]{geometry}
%\usepackage{amsmath} % for align
%\usepackage{mathtools} % for lVert
%\usepackage{graphicx} % for charts

% Set path
\newcommand{\path}{Preamble}

% Text packages
% Text packages
\usepackage[margin=1in]{geometry}                          % Margin size
\usepackage{enumerate}                                     % Custom enumeration lists
\usepackage{footmisc}                                      % Stable footnotes in headings
\usepackage{natbib}                                        % Bibliography
\usepackage{hyperref}                                      % Hyperreferences
    \hypersetup{
      colorlinks,
      citecolor=black,
      filecolor=black,
      linkcolor=black,
      urlcolor=blue
    }
\usepackage[]{authblk}                                     % Affiliation in maketitle
    \renewcommand\Affilfont{\small}                        % Small font size for affiliation
\usepackage{multicol}                                      % Multiple columns in text
\usepackage[nointegrals]{wasysym}                          % WASY2 symbols for contradiction \lightning
\usepackage{lmodern}                                       % Latin Modern font for correct T1 font encoding (no pixelated output)
\usepackage{algorithm}                                     % Pseudocode packages
\usepackage{algpseudocode}
\setlength\parindent{0pt}                                  % No indentation for the whole document

% Figure/table packages
% Figure and table packages
\usepackage{float}                                         % Figure floats
\usepackage[position=top]{subfig}                          % Subfigures
\usepackage{multirow}                                      % Merged rows in tables
\usepackage{dcolumn}                                       % Custom table delimiters
\usepackage{epstopdf}                                      % EPS files saved as PDF
\usepackage[labelfont=bf, font=normalsize]{caption}        % Bold captions
    \captionsetup{format=hang}

% Math packages
% Math packages
\usepackage{amsmath}                                       % AMS math package
\usepackage{amssymb}                                       % Math symbols
\usepackage{amsthm}                                        % Custom theorem environments
\usepackage{units}                                         % Numerical fractions
\usepackage{centernot}                                     % Logical negation in the middle of characters; e.g. not iff

% Custom theorems
% E.g. \newtheorem{command}[counter]{Display name}
\newtheorem{theorem}{Theorem}[]
\newtheorem{acknowledgement}[theorem]{Acknowledgment}
\newtheorem{axiom}[theorem]{Axiom}
\newtheorem{case}[theorem]{Case}
\newtheorem{claim}[theorem]{Claim}
\newtheorem{conclusion}[theorem]{Conclusion}
\newtheorem{condition}[theorem]{Condition}
\newtheorem{conjecture}[theorem]{Conjecture}
\newtheorem{corollary}[theorem]{Corollary}
\newtheorem{criterion}[theorem]{Criterion}
\newtheorem{exercise}[theorem]{Exercise}
\newtheorem{lemma}[theorem]{Lemma}
\newtheorem{notation}[theorem]{Notation}
\theoremstyle{definition}
\newtheorem*{definition}{Definition}
\newtheorem*{altdef}{Alternative definition}
\newtheorem{problem}[theorem]{Problem}
\newtheorem{assumption}[theorem]{Assumption}
\newtheorem{proposition}[theorem]{Proposition}
\theoremstyle{remark}
\newtheorem{example}[theorem]{Example}
\newtheorem{remark}[theorem]{Remark}
\newtheorem{solution}[theorem]{Solution}
\newtheorem{summary}[theorem]{Summary}

\renewcommand{\qedsymbol}{$\blacksquare$}                  % QED

% Mathematical operators
\DeclareMathOperator{\E}{E}                                % Expected value
\DeclareMathOperator{\var}{var}                            % Variance
\DeclareMathOperator{\cov}{cov}                            % Covariance
\DeclareMathOperator{\corr}{corr}                          % Correlation
\DeclareMathOperator{\avar}{avar}                          % Asymptotic variance
\DeclareMathOperator{\plim}{plim}                          % Probability limit
\DeclareMathOperator{\lag}{lag}                            % Lag operator
\DeclareMathOperator{\rank}{rank}                          % Rank
\DeclareMathOperator{\I}{I}                                % Identity matrix

\providecommand{\abs}[1]{\left\lvert#1\right\rvert}        % Absolute value
\providecommand{\norm}[1]{\left\lVert#1\right\rVert}       % Norm
\providecommand{\ip}[1]{\left\langle#1\right\rangle}       % Inner product
\providecommand{\csp}{\overline{\mathrm{sp}}}              % Closed span
\providecommand{\pto}{\overset{p}{\to}}                    % Convergence in probability
\providecommand{\dto}{\overset{d}{\to}}                    % Convergence in distribution

% Graphics packages
% Graphics packages
\usepackage{tikz}                                          % TikZ drawings
\usepackage{pgfplots}                                      % PGFPlots plots
    \pgfplotsset{                                          % PGFPlots options
        cartesian/.style={                                 % Declaring Cartesian plane style
            % Axis alignment
            axis x line=middle,
            axis y line=middle,
            % Axis labels alignment
            every axis x label/.style={at={(current axis.right of origin)},anchor=north west},
            every axis y label/.style={at={(current axis.above origin)},anchor=north east},
            % Legend alignment
            every axis legend/.append style={legend pos=outer north east}
        },
        % Version declaration for compatibility
        compat=1.11
    }
\usetikzlibrary{calc}
    % Node setup for extensive form games
    \tikzset{
        % Two node styles for game trees: solid and hollow
        solid node/.style={circle,draw,inner sep=1.5,fill=black},
        hollow node/.style={circle,draw,inner sep=1.5},
        % Solution nodes
        solidsol node/.style={blue,circle,draw,inner sep=1.5,fill=blue},
        hollowsol node/.style={blue,circle,draw,inner sep=1.5}
    }

\allowdisplaybreaks % so that aligned equations can be broken across pages

%% Convenient math abbreviations
\newcommand*\diff{\mathop{}\!d} % nicely formatted integral dx
%\newcommand{\E}{\mathrm{E}} % Expectation operator
\newcommand{\Var}{\mathrm{V}} % Variance operator
\newcommand{\Cov}{\mathrm{Cov}} % Covariance operator
%\newtheorem{problem}{Problem}
%\DeclarePairedDelimiter\norm{\lVert}{\rVert} % For Euclidean norm


% Title
\title{Problem Set 3 \\ \medskip \Large{Econometrics III}}
\author{\Large Jackson Bunting, Attila Gyetvai, Peter Horvath, Leonardo Salim Saker Chaves}
\date{\today}

%%%%%%%%%%%%%%%%%%%%%%%%%%%%%%%%%

\begin{document}

\maketitle
\section{Moment Estimation}

%1.1
\begin{problem} Consider an alternative proof for the asymptotics of the GMM estimator by applying the proof strategy for general M-estimators discussed in the ``Asymptotic Normality" notes. \\

\textbf{Solution:} $\hat{\theta}_n = \arg\min \bar{g}_n(\theta)^\intercal W_n \bar{g}_n(\theta) = \arg\min Q_n(\theta)$

Therefore, $\nabla_\theta Q_n(\hat{\theta}_n) = 0$. In order to take the Taylor expansion around $\theta_0$, we need to assume that $\bar{g}_n(\theta)$ is twice continuously differentiable in a neighborhood of $\theta_0$. 

So we can get:

\begin{equation*}
	\nabla_\theta Q_n(\hat{\theta}_n) = 0 = \nabla_\theta Q_n(\theta_0) +  \nabla_{\theta \theta} Q_n(\bar{\theta}) (\hat{\theta}_n - \theta_0),
\end{equation*} where

\begin{equation*}
\nabla_\theta Q_n(\theta_0) = \bar{g}_n(\theta_0)^\intercal W_n \bar{g}_n(\theta_0) = G^\intercal W \bar{g}_n(\theta_0) + o_p(1)
\end{equation*}

%\begin{equation*}
\begin{align*}
\nabla_{\theta \theta} Q_n(\bar{\theta})  & = \nabla_{\theta \theta} \bar{g}_n(\bar{\theta})^\intercal W_n\bar{g}_n(\bar{\theta}) +  \nabla_{\theta} 
\bar{g}_n(\bar{\theta})^\intercal W_n \bar{g}_n(\bar{\theta}) \\
& = \nabla_{\theta \theta} \bar{g}_n(\bar{\theta})^\intercal W_n\bar{g}_n(\theta_0) + G^\intercal W G + o_p(1) \\ 
& =O_p(1)o_p(1)+G^\intercal W G + o_p(1) \\
& = G'WG+o_p(1). 
\end{align*}
%\end{equation*}

Therefore, we can get that 

\begin{align*}
	0 = G' W \bar{g}_n(\bar{\theta}_0) + G' W G (\hat{\theta}_n - \theta_0)+o_p(1),
\end{align*}
which implies 

\begin{align*}
	\sqrt{n}(\hat{\theta}_n - \theta_0) = (G^\intercal W G)^{-1} G^\intercal W \sqrt{n} \bar{g}_n(\theta_0) \xrightarrow{d} N(0; (G^\intercal W G)^{-1} G^\intercal W S W G (G^\intercal W G)^{-1}).
\end{align*}

\end{problem}

\bigskip
%1.2
\begin{problem} Write down high-level conditions in the style of the lecture notes (or Newey-McFadden's handbook chapter). under these assumptions, derive the consistency and symptotic normality of the CUE. Verify that it attains the assymptotic variance using the optimal weighting matrix. \\
	
\textbf{Solution:} 
\begin{align*}
	\hat{\theta}_n = \arg\min\limits_{\theta \in \Theta} \bar{g}_n(\theta)^\intercal \hat{S}_n(\theta)^{-1} \bar{g}_n(\theta) = \arg\min_{\theta \in \Theta} Q_n(\theta)
\end{align*}

Obs: $\partial A^{-1} = - A^{-1} \partial A A^{-1}$, $\partial (A \cdot B) = \partial A \cdot B + A \cdot B$ and $vec(d U) = d vec(U)$. 
	
FOC: 

\begin{align*}
	\nabla_{\theta} Q_n(\bar{\theta}_n)=0,
\end{align*} 

that is,
\begin{align*}
\underbrace{\nabla_{\theta} Q_n(\bar{\theta}_n)}_{d \times 1}= \nabla_\theta \bar{g}_n(\theta)^\intercal \hat{S}_n(\theta)^{-1} \bar{g}_n(\theta) + \left(\frac{\partial}{\partial \theta} \hat{S}_n(\theta)^{-1} \bar{g}_n(\theta)    \right)^\intercal \bar{g}_n(\theta),
\end{align*} 

Obs:

\begin{align*}
	\frac{\partial}{\partial \theta} \hat{S}_n(\theta)^{-1} \bar{g}_n(\theta) & = \frac{\partial}{\partial \theta} \left(\bar{g}_n(\theta) \otimes I_n  \right) \cdot vec(\hat{S}_n(\theta)^{-1}) \\
	& = \underbrace{\hat{S}_n(\theta)^{-1} \nabla \bar{g}_n(\theta)}_{k \times d} - \underbrace{\left(\bar{g}_n(\theta)^\intercal \otimes I_k \right) \left(\hat{S}_n(\theta)^{-1} \otimes \hat{S}_n(\theta)^{-1} \right) \frac{\partial vec(\hat{S}_n(\theta))}{\partial}}_{k \times d},
\end{align*}

which implies

\begin{align*}
	0 & = \nabla_\theta \bar{g}_n(\theta)^\intercal \hat{S}_n(\theta)^{-1} \bar{g}_n(\theta)  + \nabla_\theta \bar{g}_n(\theta)^\intercal \hat{S}_n(\theta)^{-1} \bar{g}_n(\theta)  - \underbrace{\frac{\partial vec(\hat{S}_n(\theta))}{\partial \theta}^\intercal \cdot \left(\hat{S}_n(\theta)^{-1} \bar{g}_n(\hat{\theta}_n) \otimes  \hat{S}_n(\theta)^{-1} \bar{g}_n(\hat{\theta}_n) \right)}_{d \times k^2} \\
	& = 2 \nabla_\theta \bar{g}_n(\theta)^\intercal \hat{S}_n(\theta)^{-1} \bar{g}_n(\theta) - \frac{\partial vec(\hat{S}_n(\theta))}{\partial \theta}^\intercal \cdot \left(\hat{S}_n(\theta)^{-1} \bar{g}_n(\hat{\theta}_n) \otimes  \hat{S}_n(\theta)^{-1} \bar{g}_n(\hat{\theta}_n) \right)
\end{align*}

Obs: 
\begin{align*}
	\frac{\partial vec(\hat{S}_n(\theta))}{\partial \theta}^\intercal \cdot \left(\hat{S}_n(\theta)^{-1} \bar{g}_n(\hat{\theta}_n) \otimes  \hat{S}_n(\theta)^{-1} \bar{g}_n(\hat{\theta}_n)  \right) = o_p(1)
\end{align*}

If we assume $S(\cdot)$ continuous and $\sup\limits_{\theta \in \mathcal{N}} || \hat{S}_n(\theta) - S_(\theta) || \xrightarrow{p} 0$, we can conclude the above result. 

Hence, we will have something similar to the GMM proof in the notes

\begin{align*}
	0  = \nabla_\theta \bar{g}_n(\hat{\theta}_n)^\intercal \hat{S}_n(\hat{\theta}_n)^{-1} \bar{g}_n(\hat{\theta}_n) + o_p(1).
\end{align*}

For MVT,

\begin{align*}
 0 &=  	\nabla_\theta \bar{g}_n(\hat{\theta}_n)^\intercal \left(\bar{g}_n(\theta_0)+ \nabla_\theta \bar{g}_n(\bar{\theta}_n) (\hat{\theta}_n - \theta_0) \right) + o_p(1) \\
 & = (G^\intercal S^{-1} + o_p(1))\left(\bar{g}_n(\theta_0)+ G (\hat{\theta}_n - \theta_0) + o_p(1) \right) + o_p(1), 
\end{align*}

which implies
\begin{align*}
	\sqrt{n}(\hat{\theta}_n - \theta_0) = (G^\intercal S^{-1} G)^{-1} G^\intercal S^{-1} \sqrt{n} \hat{g}_n(\theta)_0 + o_p(1) \xrightarrow{d} (G^\intercal S^{-1} G)^{-1} G^\intercal S^{-1} N(0, S) \sim N(0, (G^\intercal S^{-1} G)^{-1}).
\end{align*}

\end{problem}

\bigskip
%1.3
\begin{problem}
Suppose that $\hat{\eta}_n$ is an M-estimator obtained by maximizing $\tilde{Q}_n (\eta)$.
Suppose that 
\begin{align*}
	\begin{pmatrix} \sqrt{n} \bar{g}_n (\theta_0, \eta_0) \\ \sqrt{n} \nabla_n \tilde{Q}_n (\eta_0) \end{pmatrix} \overset{d}{\to}
	N \left( 0, \begin{pmatrix} S_{gg} & S_{g \eta} \\ S_{\eta g} & S_{\eta \eta} \end{pmatrix} \right).
\end{align*}
Following the discussion above, derive the asymptotic distribution of $\sqrt{n} (\hat{\theta}_n - \theta_0)$.

\medskip
\textbf{Solution:} From before,
\begin{align*}
	\sqrt{n} \bar{g}_n (\theta_0, \hat{\eta}_n) &= \sqrt{n} \bar{g}_n (\theta_0, \eta_0) + (\Gamma + o_p(1)) \sqrt{n} (\hat{\eta}_n - \eta_0).
	\intertext{Assume that the conditions for the asymptotic normality of $\sqrt{n} (\hat{\eta}_n - \eta_0)$ hold; specifically, $\nabla_{\eta \eta} \bar{Q}_n (\eta_0) \overset{p}{\to} H(\eta_0)$. Then}
	\sqrt{n} (\hat{\eta}_n - \eta_0) &= - (H(\eta_0)^{-1} + o_p(1)) \sqrt{n} \nabla_{\eta} \bar{Q} (\eta_0) \\
	&= H(\eta_0)^{-1} \sqrt{n} \nabla_{\eta} \bar{Q} (\eta_0) + o_p(1) O_p(1).
	\intertext{Using CMT, it follows that}
	\sqrt{n} \bar{g}_n (\theta_0, \hat{\eta}_n) &= \begin{pmatrix} I_k, & \Gamma H(\eta_0)^{-1} \end{pmatrix}
	\begin{pmatrix} \sqrt{n} \bar{g}_n (\theta_0, \eta_0) \\ \sqrt{n} \nabla_{\eta} \bar{Q}_n (\eta_0) \end{pmatrix}
	+ o_p(1).
	\intertext{Consequently}
	\sqrt{n} \bar{g}_n (\theta_0, \hat{\eta}_n) &\overset{d}{\to} \begin{pmatrix} I_k, & \Gamma H(\eta_0)^{-1} \end{pmatrix} N \left( 0, \begin{pmatrix} S_{gg} & S_{g \eta} \\ S_{\eta g} & S_{\eta \eta} \end{pmatrix} \right) \\
	&\sim N \Bigg( 0, \underbrace{\begin{pmatrix} S_{gg} + \Gamma H(\eta_0)^{-1} S_{g \eta}, & S_{\eta g} + \Gamma H(\eta_0)^{-1} S_{\eta \eta} \end{pmatrix} \begin{pmatrix} I_k \\ H(\eta_0)^{-1} \Gamma^{\intercal} \end{pmatrix}}_{=: A} \Bigg).
	\intertext{Using that $S_{\eta \eta} = H(\eta_0)^{-1}$, $A$ can be written as}
	A &= S_{gg} + \Gamma H(\eta_0)^{-1} S_{g \eta} + S_{\eta g} H(\eta_0)^{-1} \Gamma^{\intercal} + \Gamma H(\eta_0)^{-3} \Gamma^{\intercal}.
	\intertext{Then, using that $\sqrt{n} (\hat{\theta}_n -\theta_0) = - ((G^{\intercal} W G)^{-1} G^{\intercal} W + o_p(1)) \sqrt{n} \bar{g}_n (\theta_0, \hat{\eta}_n)$,}
	\sqrt{n} (\hat{\theta}_n - \theta_0) &\overset{d}{\to} (G^{\intercal} W G)^{-1} G^{\intercal} W N(0,A).
	\intertext{In the special case when the first stage estimation is independent from the estimation of $\theta$, we have that $S_{g \eta} = S_{\eta g} = 0$.
	Using the optimal weighting matrix $W = S_{gg}^{-1}$, the variance matrix modifies to}
	\mathrm{Avar} (\hat{\theta}_n) &= (G^\intercal W G)^{-1} + (G^\intercal W G)^{-1} G^\intercal W \Gamma H(\eta_0)^{-3} \Gamma^{\intercal} W G (G^{\intercal} W G)^{-1}.
\end{align*}

\end{problem}

\bigskip
%1.4
\begin{problem}

\end{problem}

\bigskip
%1.5
\begin{problem}

\end{problem}

\bigskip
%1.6
\begin{problem}

\end{problem}

\bigskip
%1.7
\begin{problem}
Suppose that $\bar{g}_n (\theta, \tau) = \frac{1}{n} \sum_{t=1}^n (\theta X_t + \tau(X_t))$ and $\sup_t \E\abs{X_t} < \infty$.
Equip $\mathcal{T}$ with the uniform metric.
Show that $\bar{g}_n (\cdot)$ is s.e.
\end{problem}

\section{Empirical Process}

%2.1
\begin{problem} Show that the outer probability $\mathbb{P}^*$ is subadditive. That is, for $A \subseteq \Omega$ and $B \subseteq \Omega, \mathbb{P}^*(A \cup B) \leq \mathbb{P}^*(A) + \mathbb{P}^*(B)$. (Hint: for C = A or B, approximate $\mathbb{P}^*(C)$ with $\mathbb{P}(C')$ for some measurable C'.) \\
	
	\textbf{Solution:}  Let $A \cup B \subseteq C$ for some C. We know that $\mathbb{P}^*(C) = \inf \{\mathbb{P}(C'): C \subseteq C', \, C'$ is measurable$\}$. 
	
	$\mathbb{P}^*(A) = \inf \{\mathbb{P}(A'): A \subseteq A', \, A'$ is measurable$\}$ and $\mathbb{P}^*(B) = \inf \{\mathbb{P}(B'): B \subseteq B', \, B'$ is measurable$\}$, and $A \cup B \subseteq A' \cup B' \subseteq C'$. 
	
	$\mathbb{P}^*(A \cup B) = \inf \{\mathbb{P}(A' \cup B'): A \cup B \subseteq A' \cup B', \, A'$ and $B'$ are measurable$\} \leq \inf \{\mathbb{P}(A') +  \mathbb{P}(B'): A \subseteq A'$ and $B \subseteq B', \, A'$ and $B'$ are measurable$\}$, since $\mathbb{P}$ is subadditive. And $\inf \{\mathbb{P}(A') +  \mathbb{P}(B'): A \subseteq A'$ and $B \subseteq B', \, A'$ and $B'$ are measurable$\}$ = $\inf \{\mathbb{P}(A'): A \subseteq A', \, A'$ is measurable$\}$ + $\inf \{\mathbb{P}(B'): B \subseteq B', \, B'$ is measurable$\}$, that is, we get $\mathbb{P}^*(A \cup B) \leq \mathbb{P}^*(A) + \mathbb{P}^*(B)$.
\end{problem}

\bigskip
%2.2
\begin{problem}

\end{problem}

\bigskip
\section{Entropy Conditions}

%3.1
\begin{problem} Consider $\mathcal{C}\equiv \{(-\infty, a_1] \times (-\infty, a_2]; a_1,a_2 \in \mathbb{R}\}$. Show that the VC-index of $\mathcal{C}$ is not greater than 3.\\

\textbf{Solution:} We are considering $\chi = \mathbb{R}^2$. (BWOC) Suppose that VC-index is greater than 3, let's say 4. Then, $\mathcal{C}$ can shatter any $\chi_0 \subseteq \mathbb{R}^2$ with 4 elements. Take $X_0 = \{(1,0); (1,-1), (-1,1)\}$\\

There isn't $C \in \mathcal{C}$; $\{(1,0)\} = C \cap \chi_0$ because you will need to include $\{(1,-1)\}$ on it.\\

$\implies$ VC-index must be less than 4.
\end{problem}

\bigskip
%3.2
\begin{problem} Take this lemma (Pollard 1984, lemma 18) as given. Show that $\mathcal{C}\equiv \{\{g>0\}; g\in \mathcal{G}\}$ forms a VC-class.\\

\textbf{Solution:} Using similar steps as the lemma, we get that $\mathcal{C}' = \{\{s\in S; g(s)>0\}; g \in \mathcal{G}\}$ forms a VC-class. Then by stability properties of VC-class, $\mathcal{C}'^{c}$ will also fomr a VC-class. But notice that $\mathcal{C}'^{c} = \mathcal{C}$. So, we have the proof. 
\end{problem}

\bigskip
%3.3
\begin{problem} Use this lemma (Pollard 1984, lemma 18) to show that the collection of all closed half spaces in $\mathbb{R}^2$ (consisting of points below or above some line in$\mathbb{R}^2$) forms a VC-class.\\

\textbf{Solution:} Any line in $\mathbb{R}^2$ can be written as $0=ax-y+b$, for $a,b \in \mathbb{R}$. The points above the line will be characterized by $\{(x,y)\in \mathbb{R}^2; 0\leq ax-y+b\}$. Therefore we can write the collection of all closed half spaces as:
\begin{center}
$\mathcal{C}=\{\{(x,y)\in \mathbb{R}^2; g(x,y)\geq 0\}; g \in \mathcal{G}\}$ where \\
$\mathcal{G}=\{g:\mathbb{R}^2\mapsto \mathbb{R}; g= y-ax-b = f_1(x,y) - a f_2(x,y) - bf_3(x,y) \text{and } a,b \in \mathbb{R}\}$
\end{center}
Therefore, according to Pollard's lemma, $\mathcal{C}$ will form a VC-class with index less than 3. For the points below, $\mathcal{C}_1=\{\{(x,y)\in \mathbb{R}^2; g(x,y)\leq 0\}; g \in \mathcal{G}\}$ we can use a similar argument to show that they will form a VC-class. Finally, we can characterize the collection of all closed half spaces in $\mathbb{R}^2$ as the union of two sets, $\mathcal{C} \cup \mathcal{C}_1$. Since, each of these sets forms a VC-class, by stability property of VC-class, we have the desired result. 
\end{problem}

\bigskip
%3.4
\begin{problem} Show that $\mathcal{C}\equiv \{\text{star-shaped subsets of } \mathbb{R}^2 \ \text{that looks like } "\star" \}$ forms a VC-class.\\

\textbf{Solution:} Notice that star-shaped subsets in $\mathbb{R}^2$ can be made form the intersection of points below and above a finite number of lines. Hence, by stability of VC-class of sets, the collection of star shaped subsets in $\mathbb{R}^2$ will also form a VC-class.
\end{problem}

\bigskip
%3.5
\begin{problem} Suppose that $\mathcal{F}$ is a VC-major class and $\sup_{w \in \mathcal{W}} |f(x)| \leq K$ for some constant K and all $f \in \mathcal{G}$. Show that $\mathcal{F}$ satisfies Pollard's entropy condition.\\
\textit{(Hint: use stability properties and the above result.)}\\

\textbf{Solution:} Initially, notice that $0\leq \sup_{w \in \mathcal{W}} |f(x)/k|\leq 1$ by assumption. Based on this, we can construct functions $\tilde{f}:W\mapsto [0,1]; \tilde{f}(w) = f(w)/k$ where $f(\cdot) \in \mathcal{F}$.\\
From previous result, the class of functions $\tilde{\mathcal{F}}$ will be a VC-major class. Now, notice that any $\tilde{f}(\cdot)$ is the limit of a sequence:
\begin{center}
$\tilde{f}_m = \sum_{i=1}^m 1/m 1[\tilde{f}>i/m]$
\end{center}
Consequently, $\tilde{\mathcal{F}}$ will be the closure of the convex hull of the following:
\begin{center}
$\{1[\tilde{f}>i/m]; 1\leq i \leq m, m\geq 1, \tilde{f} \in \tilde{\mathcal{F}}\}$
\end{center}
which will form a VC-class. Hence, we can apply Theorem 2 form this lecture notes to this class of indicator functions followed by Theorem 4 also from this lecture notes in order to have $\tilde{\mathcal{F}}$ satisfying Pollard entropy condition.\\

Given that the transformation from $\mathcal{F}$ to $\tilde{\mathcal{F}}$ is just the multiplication by a constant, i.e., $(1/k)$, we can do a similar procedure considering $f = \lim_{m\to \infty} f_m$, where
\begin{center}
$f_m = k \sum_{i=1}^m 1/m 1[\tilde{f}>i/m]$
\end{center}
And we get that $\mathcal{F}$ satisfies Pollard's entropy condition.
\begin{align*}
\int_0^1 \sup_{Q \in \mathcal{Q}} \sqrt{\log(N(\varepsilon ||F||_{Q,2}, Q, \mathcal{F}))} d\varepsilon &\leq \int_0^1 k \varepsilon^{-2v/(v+2)} d\varepsilon = k \int_0^1 \varepsilon^{\frac{-2v}{v+2}} d\varepsilon \\
&< +\infty
\end{align*}
because $\frac{-2v}{v+2}>-1 \iff 2v > v+2 \iff v>2$.
\end{problem}

\bigskip
%3.6
\begin{problem}

\end{problem}

\bigskip
%3.7
\begin{problem} Let $\mathcal{V}_K$ denote functions with total variation bounded by K. It is known that if $f \in \mathcal{V}_K$, then it can be represented as $f = g - h$ for some $g, h \in \mathcal{M}_K.$ Show that $\{h(w^\intercal \beta): \beta \in \mathbb{R}^d, h \in \mathcal{V}_K \}$ satisfies Pollard's entropy condition. This class is referred to as the Type I class in Andrew's handbook chapter. \\
	
	\textbf{Solution:} We know that $h = f - g$ for some $f, g \in \mathcal{M}_K$. From problem 6, we know that $\{f(w^\intercal \beta): \beta \in \mathbb{R}^d, h \in \mathcal{M}_K \}$ and $\{g(w^\intercal \beta): \beta \in \mathbb{R}^d, h \in \mathcal{M}_K \}$ satisfy Pollard's entropy condition.  
	
	Using Lemma 3. (with the function $-1$ and $g$) and 2. (for the functions $f$ and $-g$)from the ``Empirical Processes" notes, we can get that $\mathfrak(F) \otimes \mathfrak(-G) \equiv \{f + (-g): f \in \mathcal{M}_K and -g \in \mathcal{M}_K \}$ satisfies Pollard's entropy condition with the envelope $F + (-G)$.   

\end{problem}


\end{document}
